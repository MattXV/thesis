\chapter{Conclusion}\label{ch:Conclusion}

This chapter summarises the main positive outcomes and conclusions resulting from this body of work. One can explore the overall journey, problems encountered and solutions found. Of key importance is the ``Future Work'' section highlighting how the product may be further developed with new and improved features, futher resources and time. 

\section{Discussion}
This section brings together findings from all chapters and discusses the contributions and limitations of each component of the proposed system.
\subsection{}


\section{Conclusions}

The main conclusions that may be drawn from the body of work.

\section{Future Work}

\subsection{Visual-Acoustic Mappings}
\begin{itemize}
    \item An extended Image2Reverb that integrates listener and speaker positions in the mapping process.
\end{itemize}

Fast image-to-image network to manipulate existing IRs by using depth maps. Input IRs model the acoustic phenomena of the environment. The network learns how to manipulate input frequency-dependent IRs based on a depth map: e.g., an obstacle between emitter and receiver affects high-frequency reflections.

\subsection{Materials}
\begin{itemize}
    \item A material optimiser based on ground truth RIRs and an agent tasked with painting surfaces with the correct materials.
\end{itemize}

\subsection{Perceptual Response Analysis}

\begin{itemize}
    \item Perceptual analysis of geometry reduction processes.
\end{itemize}

\subsection{Further Psychoacoustic Analysis}

\begin{itemize}
    \item Ablation study on material recognition pipelines
    \item Ablation study for the resolution of generated acoustic responses
\end{itemize}
